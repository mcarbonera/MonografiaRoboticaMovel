\chapter{Introdução}
\vspace{-2.5 cm}
% Introdução
	% Considerações iniciais
	% Objetivos
		% Objetivo Geral
		% Objetivos Específicos
	% Motivação e justificativa
	% Estrutura

Este Capítulo apresenta o que se pretende realizar com o desenvolvimento do
presente trabalho.

\section{Considerações Iniciais} 

A robótica móvel mescla aspectos das engenharias mecânica, elétrica e
eletrônica, além das ciências cognitiva, social e de computação. Os objetivos da
robótica móvel são voltados para aplicações diversas, como sensoriamento remoto,
exploração ou operação em ambientes inóspitos, carros autônomos, além de
relacionamento com humanos, em tarefas assistivas \cite{Livro_Siegwart}.

Na UTFPR câmpus Pato Branco, entre os trabalhos que tratam de robôs móveis,
destacam-se: 

a) \citeonline{Tcc_Biesek}, que desenvolveu um robô para estacionamento
automático.

b) \citeonline{Tcc_Petry}, que desenvolveu um robô seguidor de linha,
comparando uma arquitetura de controle híbrida (sistema dinâmico combinado a um
sistema a eventos discretos) com o controle por lógica \textit{Fuzzy}.

c) \citeonline{Tcc_Lopes}, que desenvolveu um robô seguidor de linha, também de
arquitetura híbrida.

d) \citeonline{Tcc_Marinho}, cujo trabalho aprimorou um robô de
sumô (robô de competição) previamente disponível. 

A execução de tarefas por robôs autônomos envolve raciocínio automatizado,
percepção e controle, que incluem problemas como o de planejamento de
trajetórias. De uma forma geral, esse problema consiste em encontrar um
trajeto que transporte o robô de uma configuração (posição e direção)
inicial até uma configuração final. A navegação é, portanto, uma importante área
de estudo em robôs autônomos \cite{mest_marchi, mest_neto, Tcc_Ottoni}.

No intuito de aumentar robustez na tarefa de navegação, pode-se usar
\textit{lógica fuzzy}, que permite lidar com incertezas do mundo real
\cite{inbook:FuzzyNavigationIntech}, já que ruídos podem provocar comportamentos
erráticos.

A lógica \textit{fuzzy} é um mapeamento não linear de dados de entrada em saídas
escalares que permite traduzir conhecimento qualitativo (linguístico) em uma
solução numérica, de modo relativamente direto \cite{Art_Mendel}. 

\subsection{Aspectos qualitativos em robótica móvel \label{SEC:SUBSUNCAO}}

Conforme \citeonline{Livro_Mataric}, controle de robôs móveis pode ser
dividido em controle deliberativo, reativo, híbrido (no sentido de
combinar os aspectos deliberativo e reativo) e baseado em comportamento.

Na arquitetura deliberativa, deve-se planejar antes de executar qualquer
ação. Essa abordagem remonta aos primeiros estudos voltados à aplicação de
conceitos da Inteligência Artificial (IA) clássica ao campo da robótica. O foco
desta abordagem é atender metas de longo prazo. Porém existe um alto custo 
computacional, já que é necessário manter uma representação interna do ambiente 
e operar buscas neste mapa, o que traz dificuldades em alcançar requisitos de curto 
prazo, dada a possibilidade iminente (imediata) de colisão. O armazenamento de mapa
é uma representação do mundo, uma forma de armazenar informações do passado. Podem ser
utilizadas matrizes, grafos. Quanto mais detalhe se armazena, maior o custo.

A arquitetura reativa, por sua vez, tem como foco uma curta escala de tempo, que
pode ser obtida pelo estabelecimento de um mapeamento direto entre entradas e
saídas, sem representação interna do ambiente. Pode ser visto como reflexo ou
instinto. 

Na arquitetura híbrida, no sentido encontrado no livro de
\citeonline{Livro_Mataric}, o controle deliberativo (``inteligente, porém
lento'') e reativo (``rápido, porém inflexível'') são combinados de modo a
extrair os pontos fortes de cada abordagem. Há a necessidade de implementar uma
camada intermediária entre os dois tipos, a fim de balancear objetivos
conflitantes (como escala de tempo). No restante deste texto, o termo
``arquitetura híbrida'' será usado apenas para designar sistemas que mesclam
aspectos de controle contínuo e a eventos discretos.  

Por fim, a arquitetura baseada em comportamento surgiu do controle reativo, diferindo
deste pelo nível de abstração. Enquanto comportamentos descrevem funções como
``seguir objeto'', ``encontrar objeto'', ``seguir parede'', reações são simples
ações como ``andar reto'' ou ``virar''. Há também uma distinção na escala de
tempo das ações, já que comportamentos não são instantâneos e operam ao longo do
tempo. 

O protótipo desenvolvido neste trabalho utiliza a arquitetura comportamental. A 
utilização de comportamentos pode acontecer por meio de duas estratégias distintas: 
arbitragem e fusão de comportamentos. Neste trabalho, o protótipo desenvolvido
explora essa arquitetura por completo, utilizando um controlador híbrido (sistemas 
contínuos com sistemas a eventos discretos) para a arbitragem de comportamentos e 
controladores \textit{fuzzy} para implementar fusão de comportamentos.  

Sobre as arquiteturas reativas e comportamentais, é importante citar a arquitetura 
de subsunção, que organiza comportamentos em camadas, de modo a estabelecer uma 
hierarquia na qual módulos de maior importância podem inibir modulos de baixa 
importância. Para exemplificar, um comportamento responsável por parar o robô deve 
inibir modulos responsáveis por seu movimento \cite{Livro_Mataric}.

\section{Objetivos}

O presente trabalho tem por objetivo assimilar aspectos relativos ao campo da
robótica móvel. Nessa seção, pretende-se delimitar quais objetivos devem ser
atendidos.

	\subsection{Objetivo Geral}
	
	Construir um robô móvel de acionamento diferencial autônomo, sem
	armazenamento de mapa, capaz de se locomover entre duas referências, desviando
	de obstáculos, se necessário. 
	
	Comparar qualitativamente as abordagens de arbitragem e fusão de comportamentos. A 
	comparação, portanto, não será em nível de tecnologia (controlador híbrido e controlador
	\textit{fuzzy}), mas em nível de arquitetura, o que faz mais sentido, já que o intuito deste
	trabalho é explorar o campo da robótica baseada em comportamentos.
 	
	\subsection{Objetivos Específicos}
 
	\begin{itemize}
	  \item Levantar o modelo matemático para o robô móvel de acionamento
	  diferencial.
	  \item Desenvolver o projeto de controle por arbitragem de comportamentos usando
	  controlador híbrido, que se trata da junção entre Sistemas Dinâmicos de Variável 
	  Contínua (SDVC) e Sistemas a Eventos Discretos (SED).
	  \item Desenvolver o projeto de controle por fusão de comportamentos usando Lógica 
	  \textit{Fuzzy}.
	  \item Construir o robô. 
	  \item Implementar os controles no sistema embarcado. 
	  \item Realizar testes de desempenho.
	  \item Comparar as estratégias de arbitragem e fusão de comportamentos. 
	\end{itemize}
	
\section{Motivação e justificativa}

Com os objetivos estipulados, pretende-se explorar o campo da robótica móvel em
seus aspectos cruciais: navegação e inteligência.

A utilização de lógica fuzzy tem por intuito simplificar o processo de
projeto, já que a introdução de variáveis linguísticas permite
resolver o problema em questão utilizando predominantemente conhecimentos
qualitativos. Deseja-se investigar tal abordagem de forma crítica, verificando
qualidades e deficiências frente à contraparte híbrida, que utiliza controle
baseado em comportamento associado a sistemas a eventos discretos.

\section{Estrutura}

Este trabalho está organizado em cinco capítulos: Introdução,
Referencial Teórico, Materiais e Método e Desenvolvimento. 

As etapas do trabalho referentes a teoria de controle e lógica \textit{fuzzy}
serão discutidas no Capítulo \ref{CAP:TEORIA} e desenvolvidas para o robô móvel 
no Capítulo \ref{CAP:DESENVOLVIMENTO}.
