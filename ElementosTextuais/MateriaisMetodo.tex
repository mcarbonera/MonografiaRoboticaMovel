\chapter{Materiais e Método}
\vspace{-2.5 cm}

Este Capítulo descreve os requisitos materiais para o desenvolvimento do
trabalho e os métodos que serão adotados para atingir os objetivos. 

\section{Materiais}

Os materiais a serem utilizados dividem-se em componentes eletrônicos e
componentes físicos. Os componentes eletrônicos podem ser vistos na tabela \ref{tab:custo}.

\input{Tabelas/TabelaCustos.tex}

Materiais para o corpo do robô, como chapa de MDF para o chassi, parafusos
de rosca, abraçadeiras não tiveram seus custos contabilizados.

Entre as ferramentas, será utilizado \textit{software} Matlab, com o
simulador Simiam \cite{Simiam}, que é implementado como um aplicativo Matlab.

\section{Metodologia}

Neste trabalho o protótipo será desenvolvido tendo como base o robô móvel de
acionamento diferencial Quickbot \cite{QuickBot}. Porém, esse robô se encontra
com alguns componentes de hardware obsoletos. Componentes mais atuais serão
utilizados.

O simulador Simiam será estudado de modo a permitir a inserção de novas
funcionalidades, como o módulo \textit{fuzzy}, pois este simulador foi
implementado visando a utilização de uma arquitetura de controle híbrida.  

Em seguida, deve-se projetar os comportamentos do robô que solucionem o problema
proposto de duas maneiras separadas: utilizando o sistema híbrido e por meio de
regras \textit{fuzzy}.

Utilizar o simulador Simiam, desenvolvido pela Universidade Georgia
Tech, para validar o algoritmo de controle.

Num próximo passo, a montagem física deve tomar espaço no cronograma,
concomitante à implementação do sistema embarcado.

Os testes serão realizados em ambiente pré-definido. Contudo, os obstáculos
serão colocados em posições distintas para verificar a robustez do algoritmo. 

Para finalizar, o robô móvel e os algoritmos implementados no sistema embarcado
serão submetidos a testes em protótipo.
