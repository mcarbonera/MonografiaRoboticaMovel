\chapter{Materiais e Método}
\vspace{-2.5 cm}

Este Capítulo descreve os requisitos materiais para o desenvolvimento do
trabalho e os métodos que serão adotados para atingir os objetivos. 

\section{Materiais}

Os materiais a serem utilizados dividem-se em componentes eletrônicos e
componentes físicos. Os componentes eletrônicos podem ser vistos na tabela \ref{tab:custo}.

\begin{table}[ht]
\caption{Tabela de custos}
\vspace{0.2 cm}
%\footnotesize
\resizebox{\linewidth}{!}{
\begin{tabular}{ll|l|l|l}
\cline{1-4}
\multicolumn{1}{|l|}{\textbf{Item}}                                                     
& \textbf{Qtd} & \textbf{Custo Unitário} & \textbf{Custo Total} &  \\
\cline{1-4}
\multicolumn{1}{|l|}{Bateria LiPo Limskey 11,1v 3S 4200mAh}                     
& 1   & 112,05              & 112,05           &  \\ \cline{1-4}
\multicolumn{1}{|l|}{Módulo para cartão de memória SD WAVGAT, comunicação SPI}                          & 1   & 2,79           & 2,79        &  \\ \cline{1-4}
\multicolumn{1}{|l|}{Ponte H dupla L298N}                                                               & 1   & 6,12           & 6,12        &  \\ \cline{1-4}
\multicolumn{1}{|l|}{Sensor infravermelho GP2Y0A21YK0F, alcance de 10 a 80 cm}                          & 5   & 13,07          & 65,35       &  \\ \cline{1-4}
\multicolumn{1}{|l|}{Módulo conversor Buck DC-DC, saída 5v}                     
& 1   & 9,14           & 9,14        &  \\ \cline{1-4}
\multicolumn{1}{|l|}{Sensor inercial IMU GY-87, com 10 graus de liberdade (MPU6050, HCM5883L, BMP180)}  & 1   & 25,9           & 25,9        &  \\ \cline{1-4}
\multicolumn{1}{|l|}{Módulo wireless NRF24L01 com antena de alcance de 1000
metros}                     & 1   & 7,69              & 7,69           &  \\
\cline{1-4}
\multicolumn{1}{|l|}{Conjuntos de motores, caixa de redução 1:34 e sensores de efeito hall (341,2 PPR)} & 2   & 47,015         & 94,03       &  \\ \cline{1-4}
\multicolumn{1}{|l|}{Esfera de rolagem}                                                                 & 1   & 2,002          & 2,002       &  \\ \cline{1-4}
\multicolumn{1}{|l|}{Tiva Connected LaunchPad TM4C1294
}& 1   & 80,00 & 80,00 &  \\ \cline{1-4} 
                                                                                                        
                                                                                                        
                                                                                                        &
                                                                                                        &
                                                                                                        Total:
                                                                                                        &
                                                                                                        405,07     &  \\ \cline{3-4}
\end{tabular}}
\label{tab:custo}
\end{table}

Materiais para o corpo do robô, como chapa de MDF para o chassi, parafusos
de rosca, abraçadeiras não tiveram seus custos contabilizados.

Entre as ferramentas, será utilizado \textit{software} Matlab, com o
simulador Simiam \cite{Simiam}, que é implementado como um aplicativo Matlab.

\section{Metodologia}

Neste trabalho o protótipo será desenvolvido tendo como base o robô móvel de
acionamento diferencial Quickbot \cite{QuickBot}. Porém, esse robô se encontra
com alguns componentes de hardware obsoletos. Componentes mais atuais serão
utilizados.

O simulador Simiam será estudado de modo a permitir a inserção de novas
funcionalidades, como o módulo \textit{fuzzy}, pois este simulador foi
implementado visando a utilização de uma arquitetura de controle híbrida.  

Em seguida, deve-se projetar os comportamentos do robô que solucionem o problema
proposto de duas maneiras separadas: utilizando o sistema híbrido e por meio de
regras \textit{fuzzy}.

Utilizar o simulador Simiam, desenvolvido pela Universidade Georgia
Tech, para validar o algoritmo de controle.

Num próximo passo, a montagem física deve tomar espaço no cronograma,
concomitante à implementação do sistema embarcado.

Os testes serão realizados em ambiente pré-definido. Contudo, os obstáculos
serão colocados em posições distintas para verificar a robustez do algoritmo. 

Para finalizar, o robô móvel e os algoritmos implementados no sistema embarcado
serão submetidos a testes em protótipo.
