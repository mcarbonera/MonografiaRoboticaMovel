\chapter{Considerações Finais}
\vspace{-2.5 cm}

Para concluir esta etapa do trabalho, algumas deliberações precisam ser feitas.
Por enquanto, uma parte do trabalho foi completamente desconsiderada: estimativa
do estado do sistema. Para simulação, muitos trabalhos acadêmicos consideram o
estado atual do robô como um valor conhecido. Para um robô real, autônomo, este
estado pode ser estimado usando técnicas como observadores, filtro de Kalman,
odometria, entre outras.

É importante salientar que há um conflito entre os objetivos deste trabalho,
já que um robô ao mesmo tempo autônomo e sem armazenamento de mapa deve possuir
a capacidade de estimar com perfeição seu estado atual. Na prática, a estimativa
de estado carrega erros que são somados no decorrer do tempo de execução. Para
corrigir tais erros, ou o robô deve armazenar mapa (representação interna do
mundo à sua volta) e reconhecer locais já visitados, ou o robô precisa de um
sistema externo (computador com câmera, sistema GPS) que envie correções
(isso reduz autonomia). 

A segunda solução pode ser adotada neste trabalho, a fim de torná-lo mais focado
no aspecto de controle e navegação, o que possibilitará explorar as diferentes
abordagens estipuladas (controle híbrido e \textit{fuzzy}) com maior
aproveitamento. 

Desta forma, uma comparação qualitativa será efetuada. Esta análise
deve ser qualitativa pois não existe um padrão de teste (\textit{benchmark})
amplamente adotado em robótica móvel. Logo, qualquer análise comparativa só pode
ser qualitativa.