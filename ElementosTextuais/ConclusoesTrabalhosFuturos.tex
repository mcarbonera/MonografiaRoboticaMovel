\chapter{Conclusões e Trabalhos Futuros}
\vspace{-2.5 cm}

Este trabalho teve por intuito defender que a definição de uma boa arquitetura para solução de um problema é
mais importante que a tecnologia empregada. Com a definição de uma arquitetura robusta, o funcionamento do sistema 
independe de tecnologia. Neste trabalho, por exemplo, utilizou-se controladores PID e controladores \textit{Fuzzy}
para implementar as estratégias de arbitragem e fusão, mas elas poderiam ter sido utilizadas em conjunto sob uma mesma
estratégia. Se um novo comportamento precisar ser implementado, qualquer tecnologia poderia ser usada para 
implementá-lo (desde soluções reativas simples até algoritmos de aprendizado de máquina), o que torna a solução
deste trabalho escalável.

Nesse sentido, é necessário comparar arquiteturas de arbitragem e fusão de comportamento. A primeira é mais escalável
que a segunda, já que cada comportamento pode ser testado de modo isolado. Na fusão de comportamentos, cada novo
comportamento adicionado se relaciona com todos os outros e a complexidade tende a aumentar de maneira exponencial.
O grande problema detectado na arquitetura de arbitragem é a situação de presença de ruído, que foi discutida na seção
dos resultados. Este trabalho adotou uma solução que provavelmente não é a melhor. Filtrar o ruído atrasa a 
``velocidade de percepção'', já que demora mais tempo para que o robô perceba que houve uma mudança no ambiente. 

Talvez exista alguma forma de detectar que o ruído ocorreu e desfazer transições indevidas. É interessante 
investigar em trabalhos futuros uma forma de tornar um autômato mais robusto na presença de ruídos. 

A arquitetura de fusão de comportamentos tem a tendência de provocar muitas oscilações, pois ao integrar 
um novo comportamento, este deve interagir com todos os outros. Em certas circunstâncias ele pode ser suprimido, em
outras, deve ``cancelar'' os outros a fim de se sobressair. As oscilações ocorrem por conta da interação entre 
comportamentos, quando está ora suprimido, ora se sobressaindo em relação aos demais.

A questão do custo computacional é outro ponto onde a arbitragem de comportamentos se sobressai,
pois apenas um comportamento é executado por vez. Desde que cada comportamento execute dentro
do tempo disponível, o custo computacional não compromete o funcionamento. Na fusão de 
comportamentos, por outro lado, todos os custos individuais se somam. 

Outro aspecto importante neste trabalho é a conversão de modelos de acionamento diferencial para uniciclo. Pode 
parecer completamente desnecessário em uma primeira leitura, porém, se houver a necessidade de controlar algum outro 
robô, com outro modelo matemático, em um mesmo ambiente, percebe-se que, desde que seja possível converter seu modelo 
para o uniciclo, o mesmo controlador deste trabalho poderá ser usado para controlá-lo.  
 
Considerando que outros robôs podem se beneficiar do mesmo algoritmo de controle deste trabalho, desde que 
seu modelo possa ser convertido para o uniciclo, seria interessante explorar essa integração em trabalhos futuros.
o robô tipo ``Segway'' seria um bom ponto de partida, já que é necessário apenas um controlador a mais para controlar
o ângulo vertical (controle de um pêndulo invertido), bem como uma forma de associar a inclinação do ``pêndulo'' com 
a velocidade linear do uniciclo.

O presente trabalho teve objetivos muito extensos e, portanto, deixou muitos pontos que são passíveis de melhoria.
Para trabalhos futuros, é interessante investigar melhorias no simulador. Acredito que seria interessante reescrevê-lo
em um ``ambiente'' de código aberto. O ``Gazebo'' do \textit{Robot Operating System} (ROS) pode ser uma boa 
alternativa. Ou talvez, escrever uma espécie de extensão para o Blender, já que o aspecto de renderização já está
implementado e seria possível obter animações muito bonitas ao concluir uma simulação.  

Acredito que melhorias futuras devem levar em conta a importância de definir uma espécie de 
``infraestrutura'' ou ``ecossistema'' para robótica, que seriam ferramentas e sistemas cujas
funcionalidades facilitem o desenvolvimento e testes de robôs. A chave para isso acredito que 
está em desenvolver um simulador mais robusto, capaz de se conectar aos robôs, usar seus
sensores como entrada para o algoritmo a ser testado e devolver o resultado a eles, para que
realizem a atuação recomendada (a Universidade Georgia Tech possui algo nesse sentido para 
os seus robôs Khepera). 

Com essa infraestrutura, é possível testar muito rapidamente quaisquer alterações no algoritmo.
Com a utilização de câmeras, a coordenada do robô real pode ser estimada, de modo a
calcular com maior precisão os erros de odometria, além de facilitar a extração de figuras e
videos para análise posterior. A partir desse ``ecossistema'', um tema mais moderno em robótica 
móvel pode ser explorado: navegação em grupos. 

%Para finalizar, gostaria de salientar que os trabalhos atuais em robótica móvel tratam de 
%temas como navegação em grupo. 

%Essa infraestrutura requer a convergência entre as áreas de Engenharia de 
%Software, Processamento de Imagens e Desenvolvimento Web.
 
%Para finalizar, gostaria de salientar que os trabalhos atuais em robótica móvel tratam de temas como navegação em 
%grupo. Dito isso, para o meu descontentamento, o presente trabalho apenas faz algo que já se fazia na década de 90. 
%As universidades que fazem o que há de mais avançado possuem uma cultura de dar continuidade a trabalhos 
%anteriores. 

%Para isso ser possível na UTFPR, é necessário desenvolver uma espécie de ``infraestrutura'' para robótica. Desenvolver
%ao longo de multiplos trabalhos acadêmicos um conjunto de ferramentas cujas funcionalidades permitiriam integrar 
%simuladores aos robôs reais. Como exemplo, posso imaginar simuladores que permitem exportar código direto para os 
%robôs . Nesse ``ecossistema'' de 
%robótica, sistemas com câmeras poderiam ajudar a desenhar o trajeto do robô e facilitar o processo de depuração.

%Para iniciar a implementação desse grande projeto, é necessário compreender o que são arquiteturas escaláveis para 
%sistemas IOT (internet das coisas). É cada vez mais comum no mundo do software falar de arquitetura 
%de microserviços, mas não está muito claro para o estudante de engenharia de computação como seria a integração das 
%tecnologias de software com dispositivos IOT. É necessário compreender a diferença entre os servidores de 
%mensageria (\textit{brokers}) que são implementados para comunicação entre microserviços e para comunicação entre 
%dispositivos IOT. Eles não são intercambiáveis (pelo menos hoje em dia).  

%É importante falar disso, pois desenvolver um ``ecossistema'' escalável para robótica traria diversas vantagens, já
%que seria mais fácil efetuar testes em robõs e facilitaria a integração de um número maior de robôs operando em um 
%mesmo ambiente. O conhecimento em nível de arquitetura contribui para formar profissionais melhores, tendo em vista 
%que o objetivo do curso de Engenharia de Computação é justamente formar profissionais capazes de integrar 
%\textit{hardware} e \textit{software}.
