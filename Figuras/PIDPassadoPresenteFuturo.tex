\pgfplotsset{
    % only effects the plot
        /tikz/fill between/every segment/.append style={
            black,    % <-- color is only applied when an empty `\addplot' argument is given
            fill opacity=0.2,
        },
        samples=100,
    }


\begin{figure}[ht]
	\centering%
	\caption{Representação gráfica da ação de um controle PID}%
	\label{fig:pidGrafico}%
	
	\begin{tikzpicture}[scale = 1, auto, node distance=2cm, on grid, >={Latex[length=2mm]}]%
		\begin{axis}[
			width=\textwidth,
			height=\axisdefaultheight,
 			xmin=-0.5,   xmax=8.5,
 			ymin=-2,   ymax=4,
 			xlabel=Tempo,
 			ylabel=Erro,
 			xtick=\empty,
 			ytick=\empty,
 			axis y line=middle,
 			axis x line=middle,
 			axis line style={{Latex[length=2mm]}-{Latex[length=2mm]}}]
 			
 			\addplot[name path = func, smooth, solid, black]
 			coordinates{ (0,1) (2,-1)(4,1)};
 			\addplot[smooth, solid, black]
 			coordinates{ (4,1) (5.3,1.5)(8,-1)};
 			
 			%\path[name path=axis] (axis cs:0,0) -- (axis cs:4,0);
 			\addplot[draw = none, name path = axis] coordinates{ (0,0) (4,0)};
 			
 			\addplot[restrict x to domain=0:4] fill between[ 
    			of = func and axis, 
  			];
 			
 			% Linhas Verticais
 			\draw[-,dashed] (axis cs:4,0)--(axis cs:4,2);
 			\draw[-,dashed] (axis cs:5,0)--(axis cs:5,2);
 			\node at (axis cs:4,-0.4) {t};
 			\node at (axis cs:5,-0.4) {$t + T_d$};
 			
 			% Setas explicativas
 			\node at (axis cs:3.5,3) {Passado};
 			\node at (axis cs:4,3.5) {Presente};
 			\node at (axis cs:4.5,3) {Futuro};
 			\draw[->] (axis cs:4,3.1)--(axis cs:4,2.2);
 			\draw[->] (axis cs:4.2,2.65)--(axis cs:5,2.65);
 			\draw[->] (axis cs:3.8,2.65)--(axis cs:3,2.65);
 			
 			% taxa de variação do meu pau na tua mão
 			\node[inner sep = 0] (inicio) at (axis cs:4,1){};
 			\node[inner sep = 0] (fim) at (axis cs:5,2.2){};
 			\draw[->] (inicio)--(fim);
		\end{axis}
	\end{tikzpicture}%
	\\
	\textbf{Fonte: baseado em \citeonline{LivroControlePID}, pg. 20}
\end{figure}