\tikzset{%
    block/.style={scale = 1.2, draw, fill=white, rectangle, 
            minimum height=2em, minimum width=3em},
    input/.style={inner sep=0pt},       
    output/.style={inner sep=0pt},      
    sum/.style = {scale = 1.2, draw, fill=white, circle, minimum size=2mm, node
    distance=1.5cm, inner sep=0pt},
    pinstyle/.style = {scale = 1.2, pin edge={to-,thin,black}}
}

\begin{figure}[h]%
    \caption{Diagrama do sistema \textit{Fuzzy} ``Seguir Parede''}%
    \centering
    
\begin{tikzpicture}[scale = 1.2, auto, node distance=2cm, on grid,
>=latex']%
	\node[block] (SE) {SE};
	\node[block, below = 1.4cm of SE] (SD) {SD};
	\node[block, below = 1.4cm of SD] (SDE) {SDE};
	\node[block, below = 1.4cm of SDE] (SDD) {SDD};
	\node[block, below = 1.4cm of SDD] (SF) {SF};
	
	\node[block, right = 6cm of SDE] (SP) {Seguir Parede};
	
	\node[block, right = 6cm of SP, yshift = 0.7cm] (SPRecY) {SPRecY};
	\node[block, above = 1.4cm of SPRecY] (SPRecX) {SPRecX};
	\node[block, below = 1.4cm of SPRecY] (RegDistYSL) {RegDistYSL};
	\node[block, below = 1.4cm of RegDistYSL] (RegDistYSD) {RegDistYSD};

	\draw [->] (SE) -- (SP);
	\draw [->] (SD) -- (SP);
	\draw [->] (SDE) -- (SP);
	\draw [->] (SDD) -- (SP);
	\draw [->] (SF) -- (SP);

	\draw [->] (SP) -- (SPRecX);
	\draw [->] (SP) -- (SPRecY);
	\draw [->] (SP) -- (RegDistYSL);
	\draw [->] (SP) -- (RegDistYSD);
\end{tikzpicture}%    
	
	\textbf{Fonte: autoria própria}
    \label{fig:fuzzyDiagramaSP}%
\end{figure}

