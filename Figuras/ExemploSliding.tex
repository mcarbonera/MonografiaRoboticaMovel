% Figura
\begin{figure}[ht]
\centering
\caption{Exemplo do modo deslizante em robótica móvel} 
	\label{fig:ExemploSliding}
	\begin{subfigure}[t]{0.5\textwidth}%
		\centering
		%\raisebox{15mm}
		
		\begin{tikzpicture}[scale = 1.1,->,auto ,node distance =4 cm and 5cm ,on grid ,
		>=latex' ,
		state/.style ={scale = 1.1,circle, draw, minimum width =0.7cm},
		finalstate/.style ={scale = 1.1,circle, draw, minimum width =0.7cm}]
			\draw[draw] (0,0) circle (10pt);
			\draw[-, dashed]  (150:3) arc(150:30:3) node[midway,left]{};		

			\node (labelA) at (0,1) {Evitar Obstáculo};
			\node (labelB) at (0,4.5) {Ir para Objetivo};
			\begin{scope}[rotate=60]
				\begin{scope}[shift={(2.1,0)}]
					\draw[->] (0, 0) -- (0.8535, 0);
				\end{scope}
			\end{scope}
			\begin{scope}[rotate=75]
				\begin{scope}[shift={(2.1,0)}]
					\draw[->] (0, 0) -- (0.8535, 0);
				\end{scope}
			\end{scope}
			\begin{scope}[rotate=90]
				\begin{scope}[shift={(2.1,0)}]
					\draw[->] (0, 0) -- (0.8535, 0);
				\end{scope}
			\end{scope}
			\begin{scope}[rotate=105]
				\begin{scope}[shift={(2.1,0)}]
					\draw[->] (0, 0) -- (0.8535, 0);
				\end{scope}
			\end{scope}
			%\begin{scope}[rotate=120]
			%	\begin{scope}[shift={(2.1,0)}]
			%		\draw[->] (0, 0) -- (0.8535, 0);
			%	\end{scope}
			%\end{scope}
			
			%\begin{scope}[rotate=123]
			%	\begin{scope}[shift={(3.8535,0)}]
			%		\begin{scope}[rotate=-123]
			%			\begin{scope}[rotate=-45]
			%				\draw[->] (0, 0) -- (0.8535, 0);
			%			\end{scope}
			%		\end{scope}
			%	\end{scope}
			%\end{scope}
			\begin{scope}[rotate=112]
				\begin{scope}[shift={(3.8035,0)}]
					\begin{scope}[rotate=-112]
						\begin{scope}[rotate=-45]
							\draw[->] (0, 0) -- (0.8535, 0);
						\end{scope}
					\end{scope}
				\end{scope}
			\end{scope}
			\begin{scope}[rotate=100]
				\begin{scope}[shift={(3.7035,0)}]
					\begin{scope}[rotate=-100]
						\begin{scope}[rotate=-45]
							\draw[->] (0, 0) -- (0.8535, 0);
						\end{scope}
					\end{scope}
				\end{scope}
			\end{scope}
			\begin{scope}[rotate=86]
				\begin{scope}[shift={(3.5535,0)}]
					\begin{scope}[rotate=-86]
						\begin{scope}[rotate=-45]
							\draw[->] (0, 0) -- (0.8535, 0);
						\end{scope}
					\end{scope}
				\end{scope}
			\end{scope}
			\begin{scope}[rotate=74]
				\begin{scope}[shift={(3.4035,0)}]
					\begin{scope}[rotate=-74]
						\begin{scope}[rotate=-45]
							\draw[->] (0, 0) -- (0.8535, 0);
						\end{scope}
					\end{scope}
				\end{scope}
			\end{scope}
		\end{tikzpicture}%
		
		\label{fig:exemplosliding2}%
		\caption{Fronteira de deslize entre comportamentos}%
		\end{subfigure}%
    	~
    	\begin{subfigure}[t]{0.5\textwidth}%
		\centering
		
		\begin{tikzpicture}[scale = 1.1,->,auto ,node distance =4 cm and 5cm ,on grid ,
		>=latex' ,
		state/.style ={scale = 1.1,circle, draw, minimum width =0.7cm},
		finalstate/.style ={scale = 1.1,circle, draw, minimum width =0.7cm}]
			\draw[draw] (0,0) circle (10pt);

			\node (labelA) at (0,1) {Evitar Obstáculo};
			\node (labelB) at (0,4.5) {Ir para Objetivo};
			\begin{scope}[shift={(0,-0.15)}]
				\draw[-, dashed]  (150:3) arc(150:30:3) node[midway,left]{};
				\begin{scope}[rotate=60]
					\begin{scope}[shift={(2.1,0)}]
						\draw[->] (0, 0) -- (0.8535, 0);
					\end{scope}
				\end{scope}
				\begin{scope}[rotate=75]
					\begin{scope}[shift={(2.1,0)}]
						\draw[->] (0, 0) -- (0.8535, 0);
					\end{scope}
				\end{scope}
				\begin{scope}[rotate=90]
					\begin{scope}[shift={(2.1,0)}]
						\draw[->] (0, 0) -- (0.8535, 0);
					\end{scope}
				\end{scope}
				\begin{scope}[rotate=105]
					\begin{scope}[shift={(2.1,0)}]
						\draw[->] (0, 0) -- (0.8535, 0);
					\end{scope}
				\end{scope}
				%\begin{scope}[rotate=120]
				%	\begin{scope}[shift={(2.1,0)}]
				%		\draw[->] (0, 0) -- (0.8535, 0);
				%	\end{scope}
				%\end{scope}
			\end{scope}
			
			\begin{scope}[shift={(0,0.15)}]
				\draw[-, dashed]  (150:3) arc(150:30:3) node[midway,left]{};
				%\begin{scope}[rotate=123]
				%	\begin{scope}[shift={(3.8535,0)}]
				%		\begin{scope}[rotate=-123]
				%			\begin{scope}[rotate=-45]
				%				\draw[->] (0, 0) -- (0.8535, 0);
				%			\end{scope}
				%		\end{scope}
				%	\end{scope}
				%\end{scope}
				\begin{scope}[rotate=112]
					\begin{scope}[shift={(3.8035,0)}]
						\begin{scope}[rotate=-112]
							\begin{scope}[rotate=-45]
								\draw[->] (0, 0) -- (0.8535, 0);
							\end{scope}
						\end{scope}
					\end{scope}
				\end{scope}
				\begin{scope}[rotate=100]
					\begin{scope}[shift={(3.7035,0)}]
						\begin{scope}[rotate=-100]
							\begin{scope}[rotate=-45]
								\draw[->] (0, 0) -- (0.8535, 0);
							\end{scope}
						\end{scope}
					\end{scope}
				\end{scope}
				\begin{scope}[rotate=86]
					\begin{scope}[shift={(3.5535,0)}]
						\begin{scope}[rotate=-86]
							\begin{scope}[rotate=-45]
								\draw[->] (0, 0) -- (0.8535, 0);
							\end{scope}
						\end{scope}
					\end{scope}
				\end{scope}
				\begin{scope}[rotate=74]
					\begin{scope}[shift={(3.4035,0)}]
						\begin{scope}[rotate=-74]
							\begin{scope}[rotate=-45]
								\draw[->] (0, 0) -- (0.8535, 0);
							\end{scope}
						\end{scope}
					\end{scope}
				\end{scope}
			\end{scope}
			
			\begin{scope}[rotate=60]
				\begin{scope}[shift={(2.9535,0)}]
					\begin{scope}[rotate=-90]
						\draw[->] (0, 0) -- (0.5535, 0);
					\end{scope}
				\end{scope}
			\end{scope}
			\begin{scope}[rotate=75]
				\begin{scope}[shift={(2.9535,0)}]
					\begin{scope}[rotate=-90]
						\draw[->] (0, 0) -- (0.5535, 0);
					\end{scope}
				\end{scope}
			\end{scope}
			\begin{scope}[rotate=90]
				\begin{scope}[shift={(2.9535,0)}]
					\begin{scope}[rotate=-90]
						\draw[->] (0, 0) -- (0.5535, 0);
					\end{scope}
				\end{scope}
			\end{scope}
			\begin{scope}[rotate=105]
				\begin{scope}[shift={(2.9535,0)}]
					\begin{scope}[rotate=-90]
						\draw[->] (0, 0) -- (0.5535, 0);
					\end{scope}
				\end{scope}
			\end{scope}
			%\begin{scope}[rotate=120]
			%	\begin{scope}[shift={(2.9535,0)}]
			%		\begin{scope}[rotate=-90]
			%			\draw[->] (0, 0) -- (0.5535, 0);
			%		\end{scope}
			%	\end{scope}
			%\end{scope}
			
			\begin{scope}[shift={(2.025,2.025)}]
				%\node[fill,circle,inner sep=1pt] at (0,0) {};
				\node (0,0) {}
   				edge[->, thick, shorten <=2pt, shorten >=2pt,bend right=45] (1,1);
			\end{scope}
   			\node (labelc) at (3.3,3.3) {Seguir Parede};
   			%edge node[above left] {Seguir Parede};
		\end{tikzpicture}%
		
		\label{fig:diagreg}%
		\caption{Resultado qualitativo de uma regularização}%
		\end{subfigure}%
		
		\textbf{Fonte: baseado em \citeonline{art:Magnus_Behavior}}
\end{figure}