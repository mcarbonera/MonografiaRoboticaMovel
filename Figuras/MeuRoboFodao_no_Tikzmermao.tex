% Define a variable as a length
\newcommand{\nvar}[2]{%
    \newlength{#1}
    \setlength{#1}{#2}
}

\def\UserDefLb{10}
\def\UserDefLe{2}
\def\UserDefbraco{5}
\def\UserDefAngulo{30}
\def\UserCoordx{5.5}
\def\UserCoordy{3.5}

\nvar{\Lb}{\UserDefLb cm} %lado base
\nvar{\Le}{\UserDefLe cm} %lado efetuador
\nvar{\braco}{\UserDefbraco cm} %lado dos braços robóticos
% Define commands for links, joints and such

\def\link{\draw [double distance=1.5mm, very thick] (0,0)--}
\def\joint{%
    \filldraw [fill=white] (0,0) circle (5pt);
    \fill[black] circle (2pt);
}

\newcommand{\triangulo}[1]
{
	\def\opa{\fpeval{sqrt(3)/2}}
	\draw (0, 0) -- +(#1, 0) -- +(0.5*#1,\opa*#1) -- +(0,0);
}

\newcommand{\base}
{
	\triangulo{\Lb}
}

\newcommand{\efetuador}[3]
{
	\begin{scope}[shift={(#1, #2)}, rotate=#3]
		\triangulo{\Le}
	\end{scope}
}

\newcommand{\DeltaPlanar}[3]{%
	%Coordenadas dos vértices da base
	\def\UserBbx{\fpeval{\UserDefLb}}
	\def\UserBby{0}
	\def\UserBcx{\fpeval{\UserDefLb/2}}
	\def\UserBcy{\fpeval{\UserDefLb*sqrt(3)/2}}
	
	%Coordenadas dos vértices do efetuador
	\def\UserEbx{\fpeval{\fpeval{\UserDefLe*cosd(#3)} + #1}}
	\def\UserEby{\fpeval{#2 + \UserDefLe*sind(#3)}}
	\def\UserEcx{\fpeval{#1 + \UserDefLe*cosd(#3+60)}}
	\def\UserEcy{\fpeval{#2 + \UserDefLe*sind(#3+60)}}
	
	%Meios vetores (entre base e efetuador)
	\def\UserVetAx{\fpeval{#1/2}}
	\def\UserVetAy{\fpeval{#2/2}}
	\def\UserVetBx{\fpeval{(\UserEbx - \UserBbx)/2}}
	\def\UserVetBy{\fpeval{(\UserEby - \UserBby)/2}}
	\def\UserVetCx{\fpeval{(\UserEcx - \UserBcx)/2}}
	\def\UserVetCy{\fpeval{(\UserEcy - \UserBcy)/2}}
	
	%Modulo dos meios vetores
	\def\UserModA{\fpeval{sqrt(\UserVetAx*\UserVetAx + \UserVetAy*\UserVetAy)}}
	\def\UserModB{\fpeval{sqrt(\UserVetBx*\UserVetBx + \UserVetBy*\UserVetBy)}}
	\def\UserModC{\fpeval{sqrt(\UserVetCx*\UserVetCx + \UserVetCy*\UserVetCy)}}

	%vetor perpendicular (unitario)
% 	\def\UserXAy{\fpeval{sqrt((\UserVetAx*\UserVetAx)/((\UservetAx*\UservetAx)+(\UservetAy*\UservetAy)))}}
% 	\def\UserXAx{\fpeval{-\UserXAy*\UserVetAy/\UservetAx}}
% 	
% 	\def\UserXBy{\fpeval{sqrt(\UserVetBx*\UserVetBx/(\UservetBx*\UservetBx +
% 	\UservetBy*\UservetBy))}}
% 	\def\UserXBx{\fpeval{-\UserXBy*\UserVetBy/\UservetBx}}
% 	
% 	\def\UserXCy{\fpeval{sqrt(\UserVetCx*\UserVetCx/(\UservetCx*\UservetCx +
% 	\UservetCy*\UservetCy))}}
% 	\def\UserXCx{\fpeval{-\UserXCy*\UserVetCy/\UservetCx}}
	
	%Modulo do vetor perpendicular (não unitário)
	\def\UserModXA{\fpeval{sqrt(\UserDefbraco*\UserDefbraco-\UserModA*\UserModA)}}
	\def\UserModXB{\fpeval{sqrt(\UserDefbraco*\UserDefbraco-\UserModB*\UserModB)}}
	\def\UserModXC{\fpeval{sqrt(\UserDefbraco*\UserDefbraco-\UserModC*\UserModC)}}
	
		%vetor perpendicular (não unitario)
	\def\UserXAy{\fpeval{-(\UserVetAx*\UserModXA)/\UserModA}}
 	\def\UserXAx{\fpeval{(\UserVetAy*\UserModXA)/\UserModA}}
 	
 	\def\UserXBy{\fpeval{-(\UserVetBx*\UserModXB)/\UserModB}}
 	\def\UserXBx{\fpeval{(\UserVetBy*\UserModXB)/\UserModB}}
 	
 	\def\UserXCy{\fpeval{-(\UserVetCx*\UserModXC)/\UserModC}}
 	\def\UserXCx{\fpeval{(\UserVetCy*\UserModXC)/\UserModC}}
	
	% Soma de vetores (braço + perpendicular)
	\def\SomavetoresAx{\fpeval{\UserVetAx + \UserXAx}}
	\def\SomavetoresAy{\fpeval{\UserVetAy + \UserXAy}}
	\def\SomavetoresBx{\fpeval{\UserVetBx + \UserXBx}}
	\def\SomavetoresBy{\fpeval{\UserVetBy + \UserXBy}}
	\def\SomavetoresCx{\fpeval{\UserVetCx + \UserXCx}}
	\def\SomavetoresCy{\fpeval{\UserVetCy + \UserXCy}}
	
	\base
	\efetuador{#1}{#2}{#3}
	
	%Adicionar braços
	%\draw [double distance=1.5mm, very thick] (0, 0) --
	%+(\SomavetoresAx,\SomavetoresAy) -- + (#1, #2);
	
	\draw [double distance=1.5mm, very thick] (0,
	 0)--(\SomavetoresAx,\SomavetoresAy);
	\draw [double distance=1.5mm, very thick] (\SomavetoresAx,\SomavetoresAy)--
	+(#1-\SomavetoresAx, #2-\SomavetoresAy);
	
	%\draw [double distance=1.5mm, very thick] (\UserBbx, \UserBby) --
	%+(\SomavetoresBx, \SomavetoresBy) -- +(\UserEbx
	%- \UserBbx,\UserEby - \UserBby);
	
	\draw [double distance=1.5mm, very thick] (\UserBbx, \UserBby) --
	+(\SomavetoresBx, \SomavetoresBy);
	\draw [double distance=1.5mm, very thick] (\UserBbx+\SomavetoresBx,
	\UserBby+\SomavetoresBy) -- +(\UserEbx-\UserBbx-\SomavetoresBx,\UserEby -
	\UserBby-\SomavetoresBy);
	
	
% 	\draw [double distance=1.5mm, very thick] (\UserBcx, \UserBcy) --
% 	+(\SomavetoresCx, \SomavetoresCy) -- +(\UserEcx
% 	- \UserBcx,\UserEcy - \UserBcy);

	\draw [double distance=1.5mm, very thick] (\UserBcx, \UserBcy) --
	+(\SomavetoresCx, \SomavetoresCy);
	
	\draw [double distance=1.5mm, very thick](\UserBcx+\SomavetoresCx,
	\UserBcy+\SomavetoresCy) -- +(\UserEcx -\UserBcx-\SomavetoresCx,\UserEcy
	-\UserBcy-\SomavetoresCy);
	
	% Adicionar juntas
	\joint
	\begin{scope}[shift={(\SomavetoresAx,\SomavetoresAy)}]
		\joint
	\end{scope}
	\begin{scope}[shift={(\UserBbx,\UserBby)}]
		\joint
		\begin{scope}[shift={(\SomavetoresBx,\SomavetoresBy)}]
			\joint
		\end{scope}
	\end{scope}
	\begin{scope}[shift={(\UserBcx,\UserBcy)}]
		\joint
		\begin{scope}[shift={(\SomavetoresCx,\SomavetoresCy)}]
			\joint
		\end{scope}
	\end{scope}
	\begin{scope}[shift={(#1,#2)}]
		\joint
	\end{scope}
	\begin{scope}[shift={(\UserEbx,\UserEby)}]
		\joint
	\end{scope}
	\begin{scope}[shift={(\UserEcx,\UserEcy)}]
		\joint
	\end{scope}
	
	% Colocar centro do efetuador
% 	\def\UserRaiz={\fpeval{sqrt(3)/3}}
% 	\def\UserPtx={\fpeval{\UserRaiz*cosd{#3+30}}}
% 	\def\UserPty={\fpeval{\UserRaiz*sind{#3+30}}}
% 	\begin{scope}[shift=(\UserPtx,\UserPty)]
% 		\fill[black] circle (2pt);
% 	\end{scope}
%\begin{scope}[shift={(\VetAx,\VetAy)}, rotate=#1]
	%\ModXC
}

\begin{figure}[h]
\centering
	\begin{tikzpicture}[scale=0.7]
		\DeltaPlanar{\UserCoordx}{\UserCoordy}{\UserDefAngulo}
	\end{tikzpicture}
\caption{Esquema de um mecanismo Delta Planar} \label{fig:RobosaoFoderoso}
\end{figure}





% %%%%%%%%%%%%%%%%%%%%%%%%%%%%%%%%%%%%%%%%%%%%%%%%%%%%%%%%%%%%%%%%%%%%%%%%%%%%%
% %%__________________________________Teste__________________________________%%
% %%%%%%%%%%%%%%%%%%%%%%%%%%%%%%%%%%%%%%%%%%%%%%%%%%%%%%%%%%%%%%%%%%%%%%%%%%%%%
% % Note. This illustration was originally made with PSTricks. Conversion to
% % PGF/TikZ was straightforward. However, I could probably have made it more
% % elegant.
% 
% Define a variable as a length
% \newcommand{\nvar}[2]{%
%     \newlength{#1}
%     \setlength{#1}{#2}
% }
% 
% % Define a few constants for drawing
% \nvar{\dg}{0.3cm}
% \def\dw{0.25}
% \def\dh{0.5}
% % Define commands for links, joints and such
% \def\link{\draw [double distance=1.5mm, very thick] (0,0)--}
% \def\joint{%
%     \filldraw [fill=white] (0,0) circle (5pt);
%     \fill[black] circle (2pt);
% }
% \def\grip{%
%     \draw[ultra thick](0cm,\dg)--(0cm,-\dg);
%     \fill (0cm, 0.5\dg)+(0cm,1.5pt) -- +(0.6\dg,0cm) -- +(0pt,-1.5pt);
%     \fill (0cm, -0.5\dg)+(0cm,1.5pt) -- +(0.6\dg,0cm) -- +(0pt,-1.5pt);
% }
% 
% \def\robotbase{%
%     \draw[rounded corners=8pt] (-\dw,-\dh)-- (-\dw, 0) --
%         (0,\dh)--(\dw,0)--(\dw,-\dh);
%     \draw (-0.5,-\dh)-- (0.5,-\dh);
%     \fill[pattern=north east lines] (-0.5,-1) rectangle (0.5,-\dh);
% }
% 
% % This macro draws a three link manipulator.
% % Input parameters:
% %   #1 theta_1
% %   #2 L_1
% %   #3 theta_2
% %   #4 L_2
% %   #5 theta_3
% %   #6 L_3
% %
% % Example:
% %   \threelink{60}{2}{-70}{2}{30}{1}
% \newcommand{\threelink}[6]{%
%     \robotbase
%     \link(#1:#2);
%    	\joint
%     \begin{scope}[shift=(#1:#2), rotate=#1]
%         \link(#3:#4);
%         \joint
%         \begin{scope}[shift=(#3:#4), rotate=#3]
%             \link(#5:#6);
%             \joint
%             \begin{scope}[shift=(#5:#6), rotate=#5]
%                 \grip
%             \end{scope}
%         \end{scope}
%     \end{scope}
% }
% 
% % %%Figura teste
% % \begin{figure}
% % \centering
% % 	\begin{tikzpicture}
% % 	    \threelink{60}{2}{-90}{2}{-60}{1}
% % 	    \begin{scope}[xshift=4cm]
% % 	        \threelink{60}{2}{-110}{2}{90}{1}
% % 	    \end{scope}
% % 	    \begin{scope}[shift={(2cm, -3.2cm)}]
% % 	        % Illustration of two different solutions to the inverse kinematic
% % 	        % problem.
% % 	        \begin{scope}[dashed]
% % 	            \threelink{-10}{2}{70}{2}{-40}{1}
% % 	        \end{scope}
% % 	        \threelink{60}{2}{-70}{2}{30}{1}
% % 	    \end{scope}
% % 	\end{tikzpicture}
% % \caption{Robô Delta Planar} \label{fig:RobosaoFoderoso}
% % \end{figure}
% 
% % \begin{figure}
% % \centering
% % 	\begin{tikzpicture}
% % 	    
% % 	\end{tikzpicture}
% % \caption{Robô Delta Planar} \label{fig:RobosaoFoderoso}
% % \end{figure}
% 
