\tikzset{%
    block/.style={draw, fill=white, rectangle, 
            minimum height=2em, minimum width=3em},
    input/.style={inner sep=0pt},       
    output/.style={inner sep=0pt},      
    sum/.style = {draw, fill=white, circle, minimum size=2mm, node distance=1.5cm, inner sep=0pt},
    pinstyle/.style = {pin edge={to-,thin,black}}
}

\begin{figure}[ht]%
    \caption{Conjuntos \textit{Fuzzy} possíveis para a variável ``altura''}%
    \label{fig:conjfuzzy}%
    \centering
    
\begin{tikzpicture}[scale = 1, auto, node distance=2cm, on grid, >=latex']%
	\begin{axis}[xmin=0.9, xmax=2.1,ymax=1.1, samples=50, xlabel={Altura (m)},
	ylabel={$\mu$}]
	
		\addplot [thick, color=black] table {
			0 1
			1 1
			1.7 0
			3 0
		};
		
		\addplot [thick, color=black] table {
			0 0
			1.3 0
			1.6 1
			1.9 0
			3 0
		};
		
		\addplot [thick, color=black] table {
			0 0 
			1.4 0
			2 1
			3 1
		};
		
  		%\addplot[blue, ultra thick] (x,x/10);
  		%\addplot[red,  ultra thick] (x,x/10 + 1);
	\end{axis}
	\node[color=gray] at (1.3,5.2) {baixo};
	\node[color=gray] at (3.3,5.2) {médio};
	\node[color=gray] at (5.7,5.2) {alto};
	
	%\draw[smooth, samples=100,domain=0:2] plot(\x,{(\x)});
\end{tikzpicture}%
	
	\textbf{Fonte: baseado em \citeonline{fuzzylilly}, p. 32.}
    %\label{fig:conjfuzzy}%
\end{figure}
