\tikzset{%
    block/.style={draw, fill=white, rectangle, 
            minimum height=2em, minimum width=3em},
    input/.style={inner sep=0pt},       
    output/.style={inner sep=0pt},      
    sum/.style = {draw, fill=white, circle, minimum size=2mm, node distance=1.5cm, inner sep=0pt},
    pinstyle/.style = {pin edge={to-,thin,black}}
}

% Figura
\begin{figure}[ht]
\centering
\caption{Operações união e intersecção em conjuntos \textit{Fuzzy}} 
	\label{fig:ANDOR}
	\begin{subfigure}[t]{0.5\textwidth}%
		\centering
		
		%%%%%%%%%%%%%%%%%%%%%%%%%%%%%%%%%%%
		\begin{tikzpicture}[scale = 1, auto, node distance=2cm, on grid, >=latex']%
	\begin{axis}[xmax=38,ymax=1.1, samples=50]
	
		\addplot[color=gray] table {
			5 0
			15 1
			25 0
		};
		
		\addplot [color=gray] table {
			15 0
			25 1
			35 0
		};
		
		%% And - intersecção
		\addplot [thick, color=black] table {
			15 0
			20 0.5
			25 0
		};
  		%\addplot[blue, ultra thick] (x,x/10);
  		%\addplot[red,  ultra thick] (x,x/10 + 1);
	\end{axis}	
	%\draw[smooth, samples=100,domain=0:2] plot(\x,{(\x)});
\end{tikzpicture}%
		%%%%%%%%%%%%%%%%%%%%%%%%%%%%%%%%%%%

		\label{fig:exemploAnd}%
		\caption{Intersecção usando função ``min''}%
		\end{subfigure}%
    	~
    	\begin{subfigure}[t]{0.5\textwidth}%
		\centering
		%%%%%%%%%%%%%%%%%%%%%%%%%%%%%%%%%%%%%%%
		\begin{tikzpicture}[scale = 1, auto, node distance=2cm, on grid, >=latex']%
	\begin{axis}[xmax=38,ymax=1.1, samples=50]
	
		\addplot [color=gray] table {
			5 0
			15 1
			25 0
		};
		
		\addplot [color=gray] table {
			15 0
			25 1
			35 0
		};
		
		%% Or - União
		\addplot [thick, color=black] table {
			5 0
			15 1
			20 0.5
			25 1
			35 0
		};
  		%\addplot[blue, ultra thick] (x,x/10);
  		%\addplot[red,  ultra thick] (x,x/10 + 1);
	\end{axis}	
	%\draw[smooth, samples=100,domain=0:2] plot(\x,{(\x)});
\end{tikzpicture}%
		%%%%%%%%%%%%%%%%%%%%%%%%%%%%%%%%%%%%%%%
		\label{fig:exemploOr}%
		\caption{União usando função ``max''}%
		\end{subfigure}%
		
		\textbf{Fonte: baseado em \citeonline{fuzzylilly}, p. 20 e 21.}
\end{figure}