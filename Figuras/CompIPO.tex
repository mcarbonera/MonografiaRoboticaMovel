\newcommand{\meuRoboLindaoCompIPO}{
	\begin{tikzpicture}[scale = 2]%
		\coordsystwo{I}
		\begin{scope}[shift={(3, 0.9)}]
			\draw[->] (0.3,0) arc (0:30:0.3);
			\draw[-] (0,0) -- (0.4,0);
			\node at (0.5,0.1) {$\phi$};
			
			% Pr 
			\begin{scope}[scale = 0.5]
				\node[color = gray] at (0.1,-0.35) {$P_r$};
			\end{scope}
			\begin{scope}[rotate=30,scale=0.5]
				\begin{scope}[scale=0.9]
					\coordsystwo{R}
				\end{scope}
				\begin{scope}[shift={(0.25,0)},rotate = -90]
					\RoboDiffClean
				\end{scope}
			\end{scope}	
		\end{scope}
		
		% Coord objetivo
		\begin{scope}[shift={(4, 3)}]
			\filldraw (0,0) circle (1pt);
			% Po
			\begin{scope}[scale = 0.5]
				\node[color = gray] at (0.2,-0.45) {$P_o$};
			\end{scope}
		\end{scope}
		
		% retas
		\node[inner sep = 3pt] (P1) at (4,3) {};
		\node[inner sep = 3pt] (P2) at (3,0.9) {};
		
		\draw[->] (0,0) -- (P1);
		\draw[->] (0,0) -- (P2);
		\draw[->] (P2) -- (P1);
		
		% angulo do objetivo
		\draw[->] (0.6,0) arc (0:37:0.6);
		\draw[-] (0,0) -- (0.4,0);
		\node at (0.8,0.4) {$\theta$};
	\end{tikzpicture}%
}%

\newcommand{\RoboDiffClean}{
	% Linhas de baixo, lat esq e dir
	\draw[-, inner sep = 0] (-0.65,-0.75) -- (0.65,-0.75);
	\draw[-, inner sep = 0] (-0.75,-0.65) -- (-0.75,0.5);
	\draw[-, inner sep = 0] (0.75,-0.65) -- (0.75,0.5);
	
	% bordas de baixo
	\draw[-, inner sep = 0] (-0.65,-0.75) -- (-0.75,-0.65);
	\draw[-, inner sep = 0] (0.65,-0.75) -- (0.75,-0.65);
	
	% Retas de cima
	\def\UserL{\fpeval{1.3/(2*cosd(45)+1)}}
	\draw[-, inner sep = 0] (-0.75,0.5) -- (\fpeval{-0.75+\UserL*cosd(45)},
	\fpeval{0.5+\UserL*sind(45)});
	
	\draw[-, inner sep = 0] (0.75,0.5) -- (\fpeval{0.75-\UserL*cosd(45)},
	\fpeval{0.5+\UserL*sind(45)});
	
	\draw[-, inner sep = 0]
	(\fpeval{-0.75+\UserL*cosd(45)},\fpeval{0.5+\UserL*sind(45)}) -- (\fpeval{0.75-\UserL*cosd(45)},
	\fpeval{0.5+\UserL*sind(45)});
	
	% Bola de rolamento
	\filldraw (0,0.45) circle (1.5pt);
	\draw (0,0.45) circle (3pt);
	
	% Desenhar rodas
	\begin{scope}[shift={(-0.75,-0.65)},rotate = 90]
		\draw[rounded corners=2pt] (0,0) rectangle ++(0.8,0.2);
	\end{scope}
	\begin{scope}[shift={(0.75,-0.65)},rotate = 90]
		\draw[rounded corners=2pt] (0,0) rectangle ++(0.8,-0.2);
	\end{scope}	
}

\begin{figure}[ht]
	\centering%
	\caption{Comportamento Ir para Objetivo}%
	\label{fig:CompIPO}%	
	\meuRoboLindaoCompIPO
	
	\textbf{Fonte: autoria própria}
\end{figure}