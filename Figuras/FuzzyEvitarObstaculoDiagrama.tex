\tikzset{%
    block/.style={scale = 1.2, draw, fill=white, rectangle, 
            minimum height=2em, minimum width=3em},
    input/.style={inner sep=0pt},       
    output/.style={inner sep=0pt},      
    sum/.style = {scale = 1.2, draw, fill=white, circle, minimum size=2mm, node
    distance=1.5cm, inner sep=0pt},
    pinstyle/.style = {scale = 1.2, pin edge={to-,thin,black}}
}

\begin{figure}[ht]%
    \caption{Diagrama do sistema \textit{Fuzzy} ``Evitar Obstáculo''}%
    \centering
    
\begin{tikzpicture}[scale = 1.2, auto, node distance=2cm, on grid,
>=latex']%
	\node[block] (SE) {SE};
	\node[block, below = 1.4cm of SE] (SD) {SD};
	\node[block, below = 1.4cm of SD] (SDE) {SDE};
	\node[block, below = 1.4cm of SDE] (SDD) {SDD};
	\node[block, below = 1.4cm of SDD] (SF) {SF};
	
	\node[block, right = 6cm of SDE] (EO) {Evitar Obstáculo};
	
	\node[block, right = 6cm of EO, yshift = 0.7cm] (RecYSDE) {RecYSDE};
	%\node[block, above = 1.4cm of RecYSDE, xshift=-0.2cm] (RecXSDE) {RecXSDE};
	\node[block, above = 1.4cm of RecYSDE] (RecYSD) {RecYSD};
	\node[block, above = 1.4cm of RecYSD] (RecYSE) {RecYSE};
	%\node[block, below = 1.4cm of RecYSDE] (RecXSDD) {RecXSDD};
	\node[block, below = 1.4cm of RecYSDE] (RecYSDD) {RecYSDD};
	\node[block, below = 1.4cm of RecYSDD] (RecXSF) {RecXSF};
	\node[block, below = 1.4cm of RecXSF] (RecV) {RecV};

	\draw [->] (SE) -- (EO);
	\draw [->] (SD) -- (EO);
	\draw [->] (SDE) -- (EO);
	\draw [->] (SDD) -- (EO);
	\draw [->] (SF) -- (EO);

	\draw [->] (EO) -- (RecYSE);
	\draw [->] (EO) -- (RecYSD);
	%\draw [->] (EO) -- (RecXSDE);
	\draw [->] (EO) -- (RecYSDE);
	%\draw [->] (EO) -- (RecXSDD);
	\draw [->] (EO) -- (RecYSDD);
	\draw [->] (EO) -- (RecXSF);
	\draw [->] (EO) -- (RecV);
\end{tikzpicture}%    
	
	\textbf{Fonte: autoria própria}
    \label{fig:fuzzyDiagramaEO}%
\end{figure}

