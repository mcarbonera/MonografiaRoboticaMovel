\tikzset{%
    block/.style={draw, fill=white, rectangle, 
            minimum height=2em, minimum width=3em},
    input/.style={inner sep=0pt},       
    output/.style={inner sep=0pt},      
    sum/.style = {draw, fill=white, circle, minimum size=2mm, node distance=1.5cm, inner sep=0pt},
    pinstyle/.style = {pin edge={to-,thin,black}}
}

% Figura
\begin{figure}[ht]
\centering
\caption{Inferência em Lógica \textit{Fuzzy}}
	\label{fig:FuzzyInferencia}
	\begin{subfigure}[b]{0.33\textwidth}%
		\centering
%		\resizebox{\linewidth}{!}{
		%%%%%%%%%%%%%%%%%%%%%%%%%%%%%%%%%%%
		\begin{tikzpicture}[scale = 0.65, auto, node distance=2cm, on grid, >=latex']%
		\begin{axis}[xmax=38,ymax=1.1, samples=50]
	
		\addplot [thick, color=gray] table {
			5 0
			15 1
			25 0
		};
		
		\addplot [thick, color=gray] table {
			15 0
			25 1
			35 0
		};
		
		\node[] at (75,95) {C};
		\node[] at (225,95) {D};
  		%\addplot[blue, ultra thick] (x,x/10);
  		%\addplot[red,  ultra thick] (x,x/10 + 1);
	\end{axis}		
		%\draw[smooth, samples=100,domain=0:2] plot(\x,{(\x)});
		\end{tikzpicture}%
		%%%%%%%%%%%%%%%%%%%%%%%%%%%%%%%%%%%
		
		\label{fig:exemploAnd}%
		\caption{Funções de pertinência para C e D}%
		\end{subfigure}%
    	~
    	\begin{subfigure}[b]{0.33\textwidth}%
		\centering
		%%%%%%%%%%%%%%%%%%%%%%%%%%%%%%%%%%%%%%%
%		\resizebox{\linewidth}{!}{
		\begin{tikzpicture}[scale = 0.65, auto, node distance=2cm, on grid, >=latex']%
	\begin{axis}[xmax=38,ymax=1.1, samples=50]
	
		\addplot [color=gray] table {
			5 0
			15 1
			25 0
		};
		
		\addplot [thick, color=black] table {
			5 0
			12.5 0.75
			17.5 0.75
			25 0
		};
	\end{axis}		
	%\draw[smooth, samples=100,domain=0:2] plot(\x,{(\x)});
\end{tikzpicture}%
%}
		%%%%%%%%%%%%%%%%%%%%%%%%%%%%%%%%%%%%%%%
		\label{fig:exemploOr}%
		\caption{Função de pertinência para regra $A \rightarrow C$, com $\mu_A =
		0,75$}%
		\end{subfigure}%
		~
		\begin{subfigure}[b]{0.33\textwidth}%
		\centering
		%%%%%%%%%%%%%%%%%%%%%%%%%%%%%%%%%%%%%%%
%		\resizebox{\linewidth}{!}{
		\begin{tikzpicture}[scale = 0.65, auto, node distance=2cm, on grid, >=latex']%
	\begin{axis}[xmin = 2, xmax=38,ymax=1.1, samples=50]
		
		\addplot [color=gray] table {
			15 0
			25 1
			35 0
		};
		
		\addplot [thick, color=black] table {
			15 0
			17.5 0.25
			32.5 0.25
			35 0
		};
		
	\end{axis}	
\end{tikzpicture}%
%}
		%%%%%%%%%%%%%%%%%%%%%%%%%%%%%%%%%%%%%%%		
		\label{fig:fig3}%
		\caption{Função de pertinência para regra $B \rightarrow D$, com $\mu_B =
		0,25$}%
		\end{subfigure}%
		
		\textbf{Fonte: baseado em \citeonline{fuzzy_passino}, p. 43 e 44.}
\end{figure}