\tikzset{%
    block/.style={draw, fill=white, rectangle, 
            minimum height=2em, minimum width=3em},
    input/.style={inner sep=0pt},       
    output/.style={inner sep=0pt},      
    sum/.style = {draw, fill=white, circle, minimum size=2mm, node distance=1.5cm, inner sep=0pt},
    pinstyle/.style = {pin edge={to-,thin,black}}
}

% Figura
\begin{figure}[ht]
\centering
\caption{Defuzzificação COG} 
	\label{fig:Defuzzificacao}
	
\begin{tikzpicture}[scale = 1, auto, node distance=2cm, on grid, >=latex']%
	\begin{axis}[xmax=38,ymax=1.1, samples=50]
	
		\addplot [thick, color=black] table {
			5 0
			12.5 0.75
			17.5 0.75
			25 0
		};
		
		\addplot [thick, color=black] table {
			15 0
			17.5 0.25
			32.5 0.25
			35 0
		};
		
		\addplot [thick, color=gray] table {
			18.1818 0
			18.1818 1
		};
		
		\node[] at (210,90) {$u_{crisp} = 18,182$};
  		%\addplot[blue, ultra thick] (x,x/10);
  		%\addplot[red,  ultra thick] (x,x/10 + 1);
	\end{axis}	
	%\draw[smooth, samples=100,domain=0:2] plot(\x,{(\x)});
\end{tikzpicture}%
		
		\textbf{Fonte: baseado em \citeonline{fuzzy_passino}, p. 47.}
\end{figure}