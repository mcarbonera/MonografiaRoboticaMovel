\begin{abstract}

Mobile robotics is an area that involves aspects of automated reasoning, perception and
control. Navigation is a crucial area of study in autonomous robots. The present work
explores the use of behavior-based architecture using two existing strategies
in the literature: arbitration and fusion. The first was implemented using
hybrid control (discrete event systems with continuous variable dynamic systems),
while the second uses \textit{Fuzzy} controllers for its execution. The prototype uses
the differential drive model and a conversion to the unicycle model was carried out in order
to make the solution scalable, since if other robots could have their models converted
to the unicycle model, the control system of this work can control them. The achieved 
results were satisfactory, however, under the arbitration strategy, it was verified
that the decision process (automaton) is sensitive to the presence of noise. Under the 
strategy of fusion, the robot showed more aggressive oscillations caused by the 
subsumption architecture.

%The present work aimed to explore the use of behavior-based architecture in the 
%field of mobile robotics. In this sense, the two existing strategies found in the literature
%were used: behavior arbitration and fusion. The first was implemented using
%hybrid control (discrete event systems with continuous variables controllers),
%while the second used \textit{Fuzzy} controllers for its execution.

\end{abstract}