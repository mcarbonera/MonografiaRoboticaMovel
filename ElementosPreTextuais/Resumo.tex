\begin{resumo}

A robótica móvel é uma área que envolve aspectos de raciocínio automatizado, percepção e 
controle. A navegação é uma área de estudo crucial em robôs autônomos. O presente trabalho 
explora o uso da arquitetura baseada em comportamento utilizando duas estratégias existentes 
na literatura: arbitragem e fusão de comportamentos. A primeira foi implementada utilizando 
controle híbrido (sistemas a eventos discretos com sistemas dinâmicos de variável contínua), 
enquanto a segunda utilizou controladores \textit{Fuzzy} para sua execução. O protótipo usa 
o modelo de acionamento diferencial e uma conversão para o modelo uniciclo foi efetuada a fim
de tornar a solução escalável, já que, se outros robôs puderem ter seus modelos convertidos 
para o modelo uniciclo, o sistema de controle deste trabalho poderá controlá-los. Os 
resultados obtidos foram satisfatórios, contudo, sob estratégia de arbitragem, verificou-se
que o processo de decisão (autômato) é sensível à presença de ruído. Sob estratégia de 
fusão, o robô apresentou oscilações mais agressivas causadas pela arquitetura de subsunção.

\end{resumo}